
%\documentclass[aps,pra,reprint]{revtex4-1}
\documentclass[twoside]{article}
\usepackage{fullpage}
%

%% Math and physics notation
\usepackage{amsmath}
\usepackage{resizegather}
\DeclareMathOperator{\acosh}{acosh}
\DeclareMathOperator{\asinh}{asinh}
\usepackage{amssymb}
\usepackage{physics}
%\newcommand{\tr}{\mathsf{Tr}}
\usepackage{braket}
\usepackage{dsfont}
\usepackage{mathtools}
\usepackage{multirow}

%% Figures
\usepackage{epstopdf}
\usepackage{float}

%% Images
\usepackage{tikz} % TikZ and PGF
\usetikzlibrary{backgrounds}
\usetikzlibrary{decorations.markings}

%% Comments
\usepackage{todonotes}
\usepackage{hyperref}
\newcommand{\Esq}{E_{\mathrm{sq}}}
\newcommand{\N}{\mathcal N}
\newcommand{\Dp}{\mathcal D_p}
\newcommand{\I}{\mathcal I}
\newcommand{\mms}{\pi}



\begin{document}
\title{Amplitude damping error correction}
\author{K. Goodenough}
%\email{kdgoodenough@gmail.com}
%\affiliation{QuTech, Delft University of Technology, Lorentzweg 1, 2628 CJ Delft, The Netherlands}

    \maketitle
\section*{Introduction}
\nonumber
The (approximate) quantum error correction code (QECC) consists of the errors
\begin{align}
A_0 = \begin{bmatrix}
1 & 0\\
0 & \sqrt{1-\gamma}
\end{bmatrix}
,~A_1 = \begin{bmatrix}
0 & \sqrt{\gamma}\\
0 & 0
\end{bmatrix}
\end{align}
where $\gamma$ is the amplitude damping parameter and the logical codewords are
\begin{align}
\Ket{\overline{0}} = \frac{1}{\sqrt{2}}\left[\Ket{0000}+\Ket{1111}\right]\\
\Ket{\overline{1}} = \frac{1}{\sqrt{2}}\left[\Ket{0011}+\Ket{1100}\right]\ .
\end{align}
The possible errors that occur with probability $\mathcal{O}(\gamma)$ are then $A_{0000}$, $A_{1000}$, $A_{0100}$, $A_{0010}$ and $A_{0001}$, i.e.~no-damping or damping on a single qubit. We see that
\begin{align}
&A_0\Ket{0} = \Ket{0},~A_0\Ket{1} = \sqrt{1-\gamma}\Ket{1}\\
&A_1\Ket{0} = 0,\ \ \ A_1\Ket{1} = \gamma\Ket{0}\\
\end{align}
so that an encoded state $\Ket{\varphi} = a\Ket{\overline{0}}+b\Ket{\overline{1}}$ gets mapped to
\begin{gather}
A_{0000}\Ket{\varphi}= \left[\frac{\Ket{0000}+\left(1-\gamma\right)^2\Ket{1111}}{\sqrt{2}}\right] + b\left[ \frac{\left(1-\gamma\right)\left(\Ket{0011}+\Ket{1100}\right)}{\sqrt{2}}\right]\\
A_{1000}\Ket{\varphi} = \sqrt{\frac{\gamma(1-\gamma)}{2}}\left[a\left(1-\gamma\right)\Ket{0111}+b\Ket{0100}\right]\\
A_{0100}\Ket{\varphi} = \sqrt{\frac{\gamma(1-\gamma)}{2}}\left[a\left(1-\gamma\right)\Ket{1011}+b\Ket{1000}\right]\\
A_{0010}\Ket{\varphi} = \sqrt{\frac{\gamma(1-\gamma)}{2}}\left[a\left(1-\gamma\right)\Ket{1101}+b\Ket{0001}\right]\\
A_{0001}\Ket{\varphi} = \sqrt{\frac{\gamma(1-\gamma)}{2}}\left[a\left(1-\gamma\right)\Ket{1110}+b\Ket{0010}\right]\\
\end{gather}
\section*{Approximate error correction}
We see that this QECC does not fulfill the standard quantum error correction conditions, in particular for the no-damping error we have

\begin{align}
\Braket{\overline{0}|A_{0000}^{\dag}A_{0000}|\overline{0}}  &= \frac{1}{2}+\frac{1}{2}\left(1-\gamma\right)^4\\
&= 1-2\gamma +3\gamma^2-2\gamma^3+\frac{1}{2}\gamma^4\\
\Braket{\overline{1}|A_{0000}^{\dag}A_{0000}|\overline{1}} &= \left(1-\gamma\right)^2\\
 & = 1-2\gamma+\gamma^2
\end{align}
and for each of the the damping errors $\tilde{A}_{1}$ we have
\begin{align}
\Braket{\overline{0}|\tilde{A}_{1}^{\dag}\tilde{A}_{1}|\overline{0}} &= \frac{\gamma\left(1-\gamma\right)}{2}\left(1-\gamma\right)^2\\
\Braket{\overline{1}|\tilde{A}_{1}^{\dag}\tilde{A}_{1}|\overline{1}} &= \frac{\gamma\left(1-\gamma\right)}{2}
\end{align}

%The QECC satisfies the quantum error correction conditions for the error-set $A_1 = \lbrace A_{1,j}\rbrace_{j=1}^4$,	

%\begin{align}
%\Braket{\overline{z}|{A}_{1,i}^{\dag}A_{1,j}|\overline{z}'} &= \alpha_{i,j} \delta_{\overline{z},\overline{z}'}\\
%&=\frac{\gamma}{2}\delta_{i,j}\delta_{\overline{z},\overline{z}'}\ .
%\end{align}
%Note that this does not correct for the error set $A_0 = \lbrace A_{0,j}\rbrace_{j=1}^4$, since 
%
%\begin{align}
%&\Braket{\overline{0}|{A}_{0,i}^{\dag}A_{0,j}|\overline{0}} = 1- \frac{\gamma}{2}\\
%\neq &\Braket{\overline{1}|{A}_{0,i}^{\dag}A_{0,j}|\overline{1}} = \left\{\begin{array}{@{}lr@{}}
%        \multirow{2}{*}{$1-\frac{\gamma}{2}$} & \text{for }i,j \in \lbrace 1,2 \rbrace \\
%                               & \text{or }i,j \in \lbrace 3,4 \rbrace \\
%        \sqrt{1-\gamma}, & \text{else}\
%        \end{array}\right\}\ .
%\end{align}
%except in the trivial case of no damping ($\gamma = 0$).

\section*{No-damping error}
The no-damping error gives the following (unnormalized) state,

\begin{gather}
a\Ket{0}+b\Ket{1}\\
\rightarrow a\left[ \frac{\Ket{0000}+\left(1-\gamma\right)^2\Ket{1111}}{\sqrt{2}}\right] + b\left[ \frac{\left(1-\gamma\right)\left(\Ket{0011}+\Ket{1100}\right)}{\sqrt{2}}\right]\ .
\end{gather}

We see that the parts of the logical zero and one are influenced unequally. If we now first measure $ZZZZ$ we can detect if we have the no-damping scenario, so that we now we have state in the form above. If we now perform a $XXXX$ rotation, we get the state

\begin{align}
a\left[ \frac{\Ket{1111}+\left(1-\gamma\right)^2\Ket{0000}}{\sqrt{2}}\right] + b\left[ \frac{\left(1-\gamma\right)\left(\Ket{0011}+\Ket{1100}\right)}{\sqrt{2}}\right]\ .
\end{align}
That is, the part of the logical one remains the same as after the no-damping error, but the part of the logical zero is different. Notice that if we continue with the ordinary QEC protocol after the $XXXX$ rotation, the no-damping error will yield the state 

\begin{gather}
a\left[ \frac{\left(1-\gamma\right)^2\Ket{0000}+\left(1-\gamma\right)^2\Ket{1111}}{\sqrt{2}}\right] + b\left[ \frac{\left(1-\gamma\right)^2\left(\Ket{0011}+\Ket{1100}\right)}{\sqrt{2}}\right]\\
\rightarrow a \Ket{\overline{0}}+b\Ket{\overline{1}}\ .
\end{gather}
What happens in the case of no-damping followed by a damping error (i.e.~an error from the set $A_1$) after the $XXXX$ rotation? Let us take damping on the first qubit $A_{1,1}$,
\begin{align}
&a\left[ \frac{\Ket{1111}+\left(1-\gamma\right)^2\Ket{0000}}{\sqrt{2}}\right] + b\left[ \frac{\left(1-\gamma\right)\left(\Ket{0011}+\Ket{1100}\right)}{\sqrt{2}}\right]\\
&\rightarrow  a\left[ \frac{(1-\gamma)\Ket{0111}}{\sqrt{2}}\right] + b\left[ \frac{\left(1-\gamma\right)\left(\Ket{0100}\right)}{\sqrt{2}}\right]\\
&\rightarrow a\Ket{0111}+b\Ket{0100}\ .
\end{align}
Similarly for the other $A_{1,j}$. Sanity check for $A_{1,3}$:
\begin{align}
&a\left[ \frac{\Ket{1111}+\left(1-\gamma\right)^2\Ket{0000}}{\sqrt{2}}\right] + b\left[ \frac{\left(1-\gamma\right)\left(\Ket{0011}+\Ket{1100}\right)}{\sqrt{2}}\right]\\
&\rightarrow  a\left[ \frac{(1-\gamma)\Ket{1101}}{\sqrt{2}}\right] + b\left[ \frac{\left(1-\gamma\right)\left(\Ket{0001}\right)}{\sqrt{2}}\right]\\
&\rightarrow a\Ket{1101}+b\Ket{0001}\ .
\end{align}
Concretely we have
\begin{align}
A_{1,1} \rightarrow a\Ket{0111}+b\Ket{0100}\\
A_{1,2} \rightarrow a\Ket{1011}+b\Ket{1000}\\
A_{1,3} \rightarrow a\Ket{1101}+b\Ket{0001}\\
A_{1,4} \rightarrow a\Ket{1110}+b\Ket{0010}\\
\end{align}

This gives for the parity checks

\begin{align}
A_{1,1} \rightarrow Z_1Z_2 = -1,~Z_3Z_4 = 1\\
A_{1,2} \rightarrow Z_1Z_2 = -1,~Z_3Z_4 = 1\\
A_{1,3} \rightarrow Z_1Z_2 = 1,~Z_3Z_4 = -1\\
A_{1,4} \rightarrow Z_1Z_2 = 1,~Z_3Z_4 = -1\\
\end{align}

We see that the first two parity checks allow us to distinguish between either $A_{1,1}$ or $A_{1,2}$ and $A_{1,3}$ or $A_{1,4}$ without distorting the codewords. In the first case, measuring $Z_1$ distinguishes between $A_{1,1}$ and $A_{1,2}$, while in the second case measuring $Z_3$ distinguishes between $A_{1,3}$ and $A_{1,4}$.\\

How do we now go from a state like $a\Ket{0111}+b\Ket{0100}$ back to $a\Ket{\overline{0}}+b\Ket{\overline{1}}$? First note that  we have that $a\Ket{0111}+b\Ket{0100} = \Ket{01}\otimes \left(a\Ket{11}+b\Ket{00}\right)$, i.e.~ the interesting stuff is in the last two qubits. We now want to have an isometry that maps
\begin{align}
\Ket{11} \rightarrow \frac{1}{\sqrt{2}}\left[\Ket{0000}+\Ket{1111}\right] = \Ket{\overline{0}}\\
\Ket{00} \rightarrow \frac{1}{\sqrt{2}}\left[\Ket{0000}+\Ket{1111}\right] = \Ket{\overline{0}}
\end{align}
But this is just the same as what needed to be accomplished in the case of no no-damping beforehand.

\newpage
\section{Symmetrized protocol}
Let 

\begin{gather}
E_{00} = A_{0000} X^{\otimes 4} A_{0000}\\
E_{01} = A_{0000} X^{\otimes 4} A_{1000}\\
E_{02} = A_{0000} X^{\otimes 4} A_{0100}\\
\vdots\\
E_{30} = A_{001	0} X^{\otimes 4} A_{0000}\\
E_{40} = A_{0001} X^{\otimes 4} A_{0000}\\
\end{gather}
These are all possible errors assuming we have at most one damping event. The (unnormalized) states corresponding to these errors with input $\Ket{\psi_c} = a\left(\frac{\Ket{0000}+\Ket{1111}}{\sqrt{2}}\right)+b\left(\frac{\Ket{0011}+\Ket{1100}}{\sqrt{2}}\right)$ are then

\begin{align}
E_{00} &\rightarrow \Ket{\xi_{00}}  = (1-\gamma')^2\Ket{\psi_c}\\
E_{01} &\rightarrow \Ket{\xi_{01}} =\frac{\sqrt{\gamma'}\left(1-\gamma'\right)^2}{\sqrt{2}}\Ket{10}\left(a\Ket{00}+b\Ket{11}\right)\\
E_{02} &\rightarrow \Ket{\xi_{02}} =\frac{\sqrt{\gamma'}\left(1-\gamma'\right)^2}{\sqrt{2}}\Ket{01}\left(a\Ket{00}+b\Ket{11}\right)\\
E_{03} &\rightarrow \Ket{\xi_{03}} =\frac{\sqrt{\gamma'}\left(1-\gamma'\right)^2}{\sqrt{2}}\left(a\Ket{00}+b\Ket{11}\right)\Ket{10}\\
E_{04} &\rightarrow \Ket{\xi_{04}} =\frac{\sqrt{\gamma'}\left(1-\gamma'\right)^2}{\sqrt{2}}\left(a\Ket{00}+b\Ket{11}\right)\Ket{01}\\
E_{10} &\rightarrow \Ket{\xi_{10}} =\frac{\sqrt{\gamma'}\left(1-\gamma'\right)^{\frac{3}{2}}}{\sqrt{2}}\Ket{01}\left(a\Ket{11}+b\Ket{00}\right)\\
E_{20} &\rightarrow \Ket{\xi_{20}} =\frac{\sqrt{\gamma'}\left(1-\gamma'\right)^{\frac{3}{2}}}{\sqrt{2}}\Ket{10}\left(a\Ket{11}+b\Ket{00}\right)\\
E_{30} &\rightarrow \Ket{\xi_{20}} =\frac{\sqrt{\gamma'}\left(1-\gamma'\right)^{\frac{3}{2}}}{\sqrt{2}}\left(a\Ket{11}+b\Ket{00}\right)\Ket{01}\\
E_{40} &\rightarrow \Ket{\xi_{20}} =\frac{\sqrt{\gamma'}\left(1-\gamma'\right)^{\frac{3}{2}}}{\sqrt{2}}\left(a\Ket{11}+b\Ket{00}\right)\Ket{10}\\
\end{align}

Note the structure of the errors. If we want to retrieve the state $a\Ket{00}+b\Ket{11}$, we don't need to distinguish between $E_{01}$ and $E_{02}$, $E_{03}$ and $E_{04}$, $E_{10}$ and $E_{20}$, $E_{30}$ and $E_{40}$. Furthermore, $E_{i0}$ and $E_{0i}$ are related by a logical $X$-flip. We then see that


\begin{center}
  \begin{tabular}{  c | c  c  c}
& $Z_1Z_2Z_3Z_4$ & $Z_1Z_2$ & $Z_3Z_4$\\
    \hline
$E_{00}$ &  0 & 0 & 0 \\
$E_{01}/E_{02}$ &  1 & 1 & 0 \\
$E_{03}/E_{04}$ &  1 & 0 & 1 \\
$E_{20}/E_{10}$ &  0 & 1 & 0 \\
$E_{30}/E_{40}$ &  0 & 0 & 1 \\
  \end{tabular}
\end{center}

Note that if $Z_1Z_2Z_3Z_4 = 1$, we only have to measure $Z_1Z_2$ to distinguish between $E_{01}/E_{02}$ and $E_{03}/E_{04}$, so in total only two measurements. For $Z_1Z_2Z_3Z_4 = 0$ and $Z_1Z_2 = 0$ we have to do a third measurement to distinguish between $E_{00}$ and $E_{30}/E_{40}$. Now we don't have distinguished the states corresponding to $E_{01}/E_{02}$ etc., so that we have an equal superposition:

\begin{align}
E_{01}/E_{02} \rightarrow \frac{\sqrt{\gamma'}\left(1-\gamma'\right)^2}{\sqrt{2}}\left(\Ket{10}+\Ket{01}\right)\left(a\Ket{00}+b\Ket{11}\right)\\
E_{03}/E_{04} \rightarrow \frac{\sqrt{\gamma'}\left(1-\gamma'\right)^2}{\sqrt{2}}\left(a\Ket{00}+b\Ket{11}\right)\left(\Ket{10}+\Ket{01}\right)\\
\end{align}

And similar for $E_{10}/E_{20}$ and $E_{30}/E_{40}$ upto a logical flip. How do we get from these states back to something proportional to $\Ket{\psi_c}$? For $E_{10}/E_{20}$ we can apply two CNOTs from the second qubit to the third and fourth qubit and then applying an $X$ flip to the first qubit.

\subsection*{Higher order effects}
We are missing the error syndromes

\begin{center}
  \begin{tabular}{  c | c  c  c}
& $Z_1Z_2Z_3Z_4$ & $Z_1Z_2$ & $Z_3Z_4$\\
    \hline
$E_{1?}$ &  1 & 0 & 0 \\
$E_{2?}$ &  1 & 1 & 1 \\
$E_{3?}$ &  0 & 1 & 1 \\
  \end{tabular}
\end{center}

\newpage
\section{Superoperator}
Define $r = \sqrt{1-\gamma}$, so that $\gamma = 1-r^2$. The superoperator is then

\begin{gather}
\begin{bmatrix}
1-5r^2+22r^2-57^3+88r^4-72r^5+28r^6-4r^7 & 0 & 0 & (1-r)^2\left(1-3r+13r^2-6r^3\right)\\
0 & \frac{1}{2}r^2\left(1+8r^2-8r^3+r^4\right) & \frac{1}{2}(1-r)^2r^2\left(1+4r-3r^2\right) & 0\\
0 & \frac{1}{2}(1-r)^2r^2\left(1+4r-3r^2\right) & \frac{1}{2}r^2\left(1+8r^2-8r^3+r^4\right) & 0\\
(1-r)^2r\left(5-12r+28r^2-20r^3+4r^4\right) & 0 & 0 & r\left(5-20r+35r^2-25r^3+6r^4\right)
\end{bmatrix}
\end{gather}
With eigenvalues

\begin{gather}
r^3 \left(2 r^3-9 r^2+9 r-1\right)\ ,\\
-r^6+r^5-r^4+r^3+r^2\ ,\\
r^2 \left(-4 r^5+28 r^4-66 r^3+63 r^2-22 r+2\right)\ ,\\
1\ ,
\end{gather}
with corresponding (unnormalized) eigenvectors

\begin{gather}
\begin{bmatrix}
0 & 1 & -1 & 0
\end{bmatrix}^T\, \\
\begin{bmatrix}
0 & 1 & 1 & 0
\end{bmatrix}^T\, \\
\begin{bmatrix}
-1 & 0 & 0 & -1
\end{bmatrix}^T\, \\
\begin{bmatrix}
\frac{-6 r^3+13 r^2-3 r+1}{4 r^5-20 r^4+28 r^3-12 r^2+5 r} & 0 & 0 & 1
\end{bmatrix}^T\ . \\
\end{gather}

\begin{figure}[h!]
    \centering
    \includegraphics[width=0.8\textwidth]{eigvals1.pdf}
\end{figure}
\end{document}